\chapter{Conclusion}
In this project, four architectures were evaluated for the forest segmentation task, with DeepLabV3+ achieving the best overall performance across Accuracy, F1-score, and AUC. Using this architecture, 12 models were trained on two different datasets: 6 on the Jordan Forest dataset and 6 on the LoveDA dataset. When evaluated on Cantabria forest images, the LoveDA-trained models demonstrated greater effectiveness in capturing fine details such as edges and small vegetation, while the Jordan-trained models proved more reliable in detecting larger forested areas.\\

The models were then applied to a larger region of Cantabria, divided into 70 tiles of 3000m x 3000m each. The resulting prediction masks were transformed into geospatial coordinates and integrated with power line data to identify potential intersection zones. During this process, some false-positive contact regions were observed, likely caused by the limited variety of training samples and the preprocessing step of resizing all input images to 512 × 512 pixels.\\

Finally, a web application was developed to integrate these functionalities, offering an interactive interface for custom image segmentation and a chatbot assistant to facilitate exploration of the results. The full code implementation is publicly available in the GitHub repository [27]. Note: The chatbot module relies on the Azure OpenAI API, meaning that an active OpenAI API endpoint must be provided in the code to enable this functionality.